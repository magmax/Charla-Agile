 \section{Técnicas}

  \subsection{Palabrotas}

  \begin{frame}
   \frametitle{Scrum, Kanban, Scrumban}
   \framesubtitle{Qué son}

   \begin{itemize}
    \item Son marcos de trabajo (\textit{frameworks}).
    \item 4 tonterías que funcionan.
    \item Evitan la mentira.
   \end{itemize}
  \end{frame}

  \begin{frame}
   \frametitle{Scrum, Kanban, Scrumban}
   \framesubtitle{Scrum en 6 palabras}

   \begin{itemize}
    \item División en \textbf{Sprints}.
    \item Reunión de \textbf{planificación} de Sprint.
    \item \textbf{Scrum} diario.
    \item Reunión de \textbf{seguimiento} de Sprint opcional.
    \item Reunión de \textbf{evaluación} de Sprint.
    \item \textbf{Panel}: Pendiente, Asignado, Terminado. Burndown.
   \end{itemize}
  \end{frame}

  \begin{frame}
   \frametitle{Scrum, Kanban, Scrumban}
   \framesubtitle{Test Driven Development}

   \begin{center}
    {\small (Foto realizada por Luismi Caballé. ¡Gracias!)}
    \includegraphics[width=8cm]{images/panelScrum.jpg}
   \end{center}
  \end{frame}

  \begin{frame}
   \frametitle{Scrum, Kanban, Scrumban}
   \framesubtitle{Aplicaciones: Kunagi}

   \begin{center}
    \only<1>{
    \includegraphics[width=11cm]{images/kunagiwhiteboard.png}
    }
    \only<2>{
    \includegraphics[width=11cm]{images/kunagidashboard.png}
    }
   \end{center}

  \end{frame}

  \begin{frame}
   \frametitle{Scrum, Kanban, Scrumban}
   \framesubtitle{Aplicaciones: Agilefant}

   FIXME: Foto Burndown Agilefant

  \end{frame}

  \begin{frame}
   \frametitle{Scrum, Kanban, Scrumban}
   \framesubtitle{Agilefant}

   FIXME: Foto Backlog Agilefant

  \end{frame}

  \begin{frame}
   \frametitle{Pomodoro}
   \framesubtitle{Agilefant}

   \begin{columns}
    \begin{column}{4cm}
     \includegraphics[width=4cm]{images/pomodoro.png}
    \end{column}
    \begin{column}{7cm}
     \begin{itemize}
      \item 25 minutos de máxima concentración.
      \item 5 minutos descanso.
      \item 2 listas.
      \item Marcas.
     \end{itemize}
    \end{column}
   \end{columns}
  \end{frame}

  \subsection{TDD}

  \begin{frame}
   \frametitle{Mediciones}
   \framesubtitle{¿Qué es el código limpio?}

   \begin{center}
    \includegraphics[width=7cm]{images/wtfm.jpg}
   \end{center}
  \end{frame}



  \begin{frame}
   \frametitle{Test Driven Development}
   \framesubtitle{Qué significa}

   Pasos:
   \begin{enumerate}
    \item Escribir el \textbf{test}.
    \item Escribir el \textbf{código mínimo} que pasa el test.
    \item \textbf{Refactorizar}.
   \end{enumerate}
  \end{frame}


  \begin{frame}
   \frametitle{Código limpio}
   \framesubtitle{Importancia}

   \begin{columns}
    \begin{column}{4cm}
     \includegraphics[width=4cm]{images/cleancode.png}
    \end{column}
    \begin{column}{7cm}
     \begin{itemize}
      \item El código se escribe 1 vez, pero se lee muchas.
      \item Las pruebas pueden ser la mejor documentación.
      \item Tanto las pruebas como el código nos cuentan una historia.
     \end{itemize}
    \end{column}
   \end{columns}
  \end{frame}


  \begin{frame}
   \frametitle{Código limpio}
   \framesubtitle{Importancia de las pruebas}

   Escribir pruebas \ldots
   \begin{itemize}
    \item \ldots nos ayudará a escribir mejores pruebas.
    \item \ldots nos obligará a escribir código fácil de probar.
    \item \ldots mejorará la comprensión y usabilidad de nuestro código.
   \end{itemize}
  \end{frame}

  \begin{frame}
   \begin{center}
    \includegraphics[width=10cm]{images/profesional.jpg}
   \end{center}
  \end{frame}

 \begin{frame}
  \frametitle{Profesionales}
  \framesubtitle{Ward Cunningham}
  \begin{columns}
   \begin{column}{6cm}
    \only<1>{
    \begin{itemize}
     \item Creador del ``\textit{wiki}''.
     \item Representante de XP.
     \item Trabajó para Microsoft en el ``grupo prácticas y patrones''
     \item Director del ``Committer Community Development'' de la
	   Fundación Eclipse.
    \end{itemize}
    }
    \only<2>{
    ``You know you are working on
    clean code when each routine
    you read turns out to be pretty
    much what you expected.''
    }
    \only<3>{
    ``You can call it beautiful code
    when the code also makes it
    look like the language was
    made for the problem''
    }
   \end{column}
   \begin{column}{5cm}
    \includegraphics[width=5cm]{images/mentorsCunningham.jpg}
   \end{column}
  \end{columns}
 \end{frame}


 \begin{frame}
  \frametitle{Profesionales}
  \framesubtitle{Kent Beck}
  \begin{columns}
   \begin{column}{4cm}
    \includegraphics[width=4cm]{images/mentorsKentBeck.jpg}
   \end{column}
   \begin{column}{6cm}
    \only<1>{
    \begin{itemize}
     \item Creador de XP.
     \item Creador de metodologías TDD.
     \item Creador de JUnit con Erich Gamma.
    \end{itemize}
    }
    \only<2>{
    Rules for simple design:\\
    1. Appropriate for the intended audience\\
    2. Every idea communicated\\
    3. Factored. (no duplication)\\
    4. Minimal. (fewest elements)
    }
   \end{column}
  \end{columns}
 \end{frame}

  {
   \usebackgroundtemplate{\includegraphics[width=\paperwidth]{images/ventanaRota.jpg}}

  \begin{frame}
   \frametitle{Código limpio}
   \framesubtitle{Teoría de la ventana rota}
   .%sí, esto es cutre, pero no importa demasiado.
   \\[6pc]
   El caos comienza por una ventana rota.

  \end{frame}
  }

  {
   \usebackgroundtemplate{}

  \begin{frame}
   \frametitle{Código limpio}
   \framesubtitle{Deuda técnica}

   \begin{columns}
    \begin{column}{4cm}
     \includegraphics[width=4cm]{images/domino.jpg}
     \end{column}
    \begin{column}{7cm}
     La ``\textbf{Deuda técnica}'' es como saltarse una pieza:
     \begin{itemize}
      \item Iremos más rápido.
      \item El programa funcionará.
      \item Tarde o temprano ocurrirá un problema y tendremos que
	    averiguar por qué ha ocurrido.
     \end{itemize}
    \end{column}
   \end{columns}
  \end{frame}
  }
